\documentclass[letterpaper,twocolumn,11pt]{article}

\usepackage[utf8]{inputenc}
\usepackage[T1]{fontenc}
\usepackage{fouriernc}
\usepackage{amsmath}
\usepackage{amssymb}
\usepackage{amsthm}
\usepackage{booktabs}
\usepackage[labelfont=bf]{caption}
\usepackage{enumitem}
\usepackage{float}
\usepackage{geometry}
\usepackage{graphicx}
\usepackage{mathtools}
\usepackage{microtype}
\usepackage{nicefrac}
\usepackage{physics}
\usepackage{siunitx}
\usepackage{xcolor}

\geometry{margin=1in}

\newcommand{\refdes}[1]{\texttt{#1}}
\newcommand{\wtemp}{\ensuremath{\blacklozenge}}

\title{Lab Power Supply Manual}
\author{Alex Striff}
\date{\today}

\begin{document}
\maketitle
\tableofcontents
\thispagestyle{empty}

\section{Acknowledgments}

The creation of this lab power supply was generously funded by the Reed College
Physics Department. I would like to thank Edgar Perez for his kind advice, as
well as Lucas Illing for supporting this project.

\section{Motivation}

Often when working in electronics, several voltage sources are needed. In
addition to a main power rail, one may need a complementary negative power rail
for analog circuitry, or a different logic power supply at \SI{3.3}{\V} instead
of \SI{5}{\V}. Low-impedance bias voltages are also a frequent requirement,
needed for biasing BJTs, comparator inputs, and more.

For most applications, a power supply fulfilling the requirements of these
applications need not be capable of supplying much more than \SI{1}{\A} of
current, must have a stable voltage output (requiring a linear regulator), and
must have a current limiting function.  Additionally, it would be nice if the
power supply was much smaller than a conventional \SI{30}{\V}, \SI{3}{\A} output
bench power supply with a large transformer, being conveniently powered from a
common wall plug, batteries, or any other DC power source at hand. I have
attempted to construct such a power supply.

\section{Safety Notice (Grounding)}

Please note that the negative voltage output of the supply is directly connected
to the negative voltage input to the supply. That is, the output voltage of the
supply floating relative to earth ground if and only if the input voltage is
floating.

If you are uncertain if the supply (or any other piece of equipment) is
floating, it is quick and simple to check if this is the case. Set a digital
multimeter to its resistance measurement or continuity check mode. Connect one
probe to the negative output of the PSU (the black five-way binding post) or to
the other terminal in question, and then connect the other probe to earth
ground. This can be done either by insertion in to the \emph{earth} socket of a
wall outlet, or by touching the outside of a BNC connector on any nearby
oscilloscopes, as these are almost always earth grounded. If inserting into a
wall outlet, be certain that you know which hole is which, and that you are
using a multimeter approved for wall testing (most are).

If the output is not floating (earth-referenced), then one must be careful to
only connect oscilloscope ground leads to the negative output of the supply.
Whatever the ground lead is connected to will be shorted to earth ground. If an
incorrect connection is made, then connected circuit components, oscilloscopes,
or computers (\textit{e.g.} through USB) may be damaged. If you are uncertain,
connect probes as if the circuit is not floating.

If the output is floating (or if you know which terminals are earthed and which
are not), then the supply may be safely connected to other voltage sources in
whatever configurations are convenient, such as in a dual-rail setup.

\section{Usage Instructions}

\subsection{Setting the output voltage}

Connect a voltmeter to the output of the supply. You may use either the binding
posts or the test points labeled on the board to achieve this. Turn the PSU on
and adjust the voltage using the \emph{coarse} and \emph{fine} adjustment knobs
as needed. Note that the maximum output voltage is about \SI{1.5}{\V} below the
input voltage.

\subsection{Setting the current limit}\label{sec:set_limit}

To set the current limit to its minimum value, \emph{short} the minimum limit
jumper (labeled \emph{Min Lim}, \refdes{JP1}). To set the current limit to
higher than this value, the jumper must be \emph{open}.

The current limit is set to fixed values by moving the switches on the
board. The default values are 1, 2.5, 5, 10, 25, 50, and \SI{100}{\mA}. To set
the limit to any of the upper four values, the switch for the lower values must
be in its rightmost position, as indicated on the board.

Alternatively, the switch section on the board may not be populated, and a
\SI{10}{\kohm} (preferably 10-turn) potentiometer (\refdes{RV4}) may be
soldered in to provide a continuously variable current limit.

To set the current limit to its maximum value, \emph{open} the no limit jumper
(labeled \emph{No Limit}, \refdes{JP2}). To set the current limit to lower than
this value, the jumper must be \emph{shorted}. The maximum value is about
\SI{2.0}{\A} in normal operation, and less when the device shuts down to prevent
overheating.

\subsection{Indicator LEDs}

The power supply includes two indicator LEDs for when output voltage regulation
is not guaranteed.

The \textit{Hot} indicator LED (\refdes{D2}) lights when the main regulation IC
(see Section~\ref{sec:lt3081}) starts to get hot (when the junction temperature
is about \SI{100}{\celsius} or above). This light is a warning, and the output
should continue to be regulated as normal. If the IC continues to heat up (to a
junction temperature of \SI{125}{\celsius}), then internal protection circuitry
will prevent damage and reduce the output voltage.

The \textit{$V_\text{out}$ Error ($I$ lim)} indicator LED (\refdes{D1}) lights
when the actual output voltage is not sufficiently close to the set output
voltage. The most common cause for this is if the current limiting function is
active, but internal protection circuitry or a set voltage that is too high may
also cause an output error.

\subsection{Monitoring the output current}

If it is not preferred to use an ammeter to measure the output current, the
\textit{$I_\text{out}$} test point is provided for convenient measurement or
external control of the load current. The signal at \textit{$I_\text{out}$}
is one volt for every ampere of output current, \emph{including the internal
load} (see Sections~\ref{sec:chars} and~\ref{sec:int_load}). For example, if the
supply is outputting \SI{25}{\mA} total, then \textit{$I_\text{out}$} should
read \SI{25}{\mV}.

\subsection{Monitoring the internal temperature}

For more quantitative information about the temperature of the main regulation
IC (see Section~\ref{sec:lt3081}) than is provided by the \textit{Hot} indicator
LED, the \textit{Temp} test point is provided. The signal at \textit{Temp} is
one millivolt for every degree Celsius of junction temperature. For example, if
the junction temperature of the IC is about \SI{73}{\celsius} (subject to
variation inside the IC), then \textit{Temp} should read \SI{73}{\mV}.

\subsection{Calibrating the current limit}

The power supply requires a minimum load in order to regulate the output voltage
properly. An internal load usually supplies this minimum load (see
Section~\ref{sec:int_load}), but this offsets the effective current limit on the
output. A trimmer potentiometer is provided to compensate for this offset.

Set the supply to the minimum limit as described in Section~\ref{sec:set_limit}.
Using a screwdriver, adjust the potentiometer (labeled \emph{Load
Offset}, \refdes{RV1}) until the output voltage is stable (no current limiting),
and then carefully reverse direction and adjust until the current limit just
starts to activate. This may be judged by checking the output voltage with a
voltmeter, or by using the built-in indicator if it is calibrated as described
in Section~\ref{sec:cal_offset}

\subsection{Calibrating the output error indicator}\label{sec:cal_offset}

The issues associated with creating a reliable output error indicator are
discussed in Section~\ref{sec:how_offset}. If necessary, a trimmer potentiometer
(\refdes{RV5}) may be populated to correct the default setting by the resistor
\refdes{R21}.

Attach a voltmeter to the PSU and set the output voltage to \SI{1}{\V}. Attach
an external potentiometer to the output of the supply, valued to draw a typical
current for your application. Set the PSU current limit so that increasing the
load current will trigger the limit function. If the \SI{1}{\V} output does not
demand enough current, it may be increased, but try to keep it as low as
possible (see \ref{sec:how_offset} for why). Wait until the temperature of the
circuit has stabilized. Error on the side of drawing a slightly lower current
than needed, depending on the sensitivity required (see below). Increase the
load gradually, and watch the error indicator LED (\refdes{D1}).

If the default indication threshold set by \refdes{R21} is not sensitive enough
for your needs, solder in the \SI{500}{\kohm} \refdes{RV5} and try the
adjustment procedure below.

If this does not work, or if the default indication did not work at all, solder
in \refdes{RV5} and \emph{remove} \refdes{R21}. This provides a wider range of
variation for the indication threshold, at the cost of a coarser adjustment
rate.

With the potentiometer \refdes{RV5} and the default resistor \refdes{R21}
soldered on the board or not depending on your needs, adjust the external load
potentiometer until the output is 2 -- \SI{10}{\mV} below the set voltage. If
you intend to use the supply \emph{only} above about \SI{5}{\V}, the lower the
better (you may even be able to remove \refdes{RV5} and wire a short across
\refdes{R21} for a bit more sensitivity). To complete the calibration, adjust
\refdes{RV5} to the barrier where \refdes{D1} just barely lights, or perhaps
flickers.

\section{Electrical Characteristics}\label{sec:chars}

\begin{table*}[ht]
  \centering
  \caption{Electrical Characteristics. The \wtemp\ mark indicates specifications
  which apply over the full operating temperature range. Otherwise,
specifications are at (junction) temperatures of \SI{25}{\celsius}.}
  \begin{tabular}{lr|lr|cccr}
    \toprule
    \multicolumn{2}{c}{\textbf{Parameter}} &
    \multicolumn{2}{c}{\textbf{Conditions}} & \textbf{Min} & \textbf{Typ} &
    \textbf{Max} & \multicolumn{1}{c}{\textbf{Units}} \\
    \midrule
    Input Voltage & $V_\text{in}$ & & \wtemp & 5 & & 32 & \si{\V} \\
    Input Voltage & $V_\text{in}$ & & \wtemp & 5 & & 32 & \si{\V} \\
    Input Voltage & $V_\text{in}$ & & \wtemp & 5 & & 32 & \si{\V} \\
    \bottomrule
  \end{tabular}
  \label{tab:chars}
\end{table*}

\section{Principles of Operation}

\subsection{The LT3081 linear regulator}\label{sec:lt3081}

\subsection{The internal load}\label{sec:int_load}

\subsection{The $V_\text{out}$ error indicator circuit}\label{sec:how_offset}

\section{Schematic Diagram}

\section{Printed Circuit Board Footprints}

\section{Bill of Materials}

\section{Colophon}

The schematic diagram and printed circuit board layout were both created with
KiCAD. This manual was compiled using \LaTeX. Relatively fine (\SI{0.51}{\mm})
lead-based solder and mildly activated rosin flux were required to solder the
components to the PCB.

\end{document}

